\documentclass{article}
\usepackage{amsmath,amssymb}
\usepackage{color}
\usepackage{bm}
\usepackage[sort&compress]{natbib}
\usepackage[normalem]{ulem}	% Part of the standard distribution


\begin{document}
\section{Reviewer I}

\textbf{Question:}
\begin{quotation}
The authors found that quite surprisingly at high viscosity ratio, the results are
independent of any inertial effects. It is known that Lattice Boltzmann has some problem for high
viscosity ratio binary fluids. It will be better if the authors provide some details how the high
viscosity ratio is tackled in their LB code.
\end{quotation}

\textbf{Answer:} The reviewer is right that accuracy of the method deteriorate with the high
viscosity ratio. However, the parameters we chose here as $\tau_{\mathrm{liq}}=3.5-7.5$ and with
grids in the vertical direction as $N_y=202$ is sufficient to suppress effects. We have added all
the necessary references in the text, lines $481-491$.

\textbf{Question:}
\begin{quotation}
Figure 8 needs to redrawn. Also, there are some inconsistencies in other figures. Authors may
want to review all the figures and ensure that they have uniform fonts, etc.
\end{quotation}

\textbf{Answer:} We redrew all the figures. For the figure 8 the data was extracted from the
figure and rescaled for the comparison reasons.

\section{Reviewer II}

\textbf{Question:}
\begin{quotation}
The main quantity that the paper focuses on is the thickness of the film between the bubbles and the
wall. As shown by Figure 8 and as mentioned by the Authors in Section 4, the film thickness varies
along the bubble. However, the Authors do not explain the choice of the location of the measurement
of the film thickness. This choice is of primary importance when comparing results with those of
the
literature and in particular with works on finger propagation in which the film thickness is
measured
”at infinity”.
\end{quotation}

\textbf{Answer:} The explanations are added in lines 434-453.
 
\textbf{Question:}
\begin{quotation}
It may be appropriate to introduce Figure 8 on thickness variation before the main results on film
thickness.
\end{quotation}

\textbf{Answer:} We agree and moved the section before the results on the capillary number
dependency.


\textbf{Question:}
\begin{quotation}
The lines 54-55 indicating that ”numerical simulations and experimental studies showed [...]
Reynolds
number effects on the film thickness for capillary numbers larger than 0.003” is in contradiction
with
the lines 32-33 which state ”negligible Reynolds number effects on the film thickness for a
relatively
wide range of Reynolds numbers.”
\end{quotation}

\textbf{Answer:} We elaborate the Reynolds number influence in lines 34-43.

\textbf{Question:}
\begin{quotation}
Although inertia has been shown to have a minor effect on film thickness, inertia does influence
some flow variables in the Bretherton’s problem (see Ref. 3 for instance). To check whether inertia
is important or not, and to discuss the assumption of uniform density, the Authors should provide
the reader with the Reynolds numbers associated with each simulation. This is also required for the
consistency of Figure 7 in which results are compared with other works of the literature.
\end{quotation}

\textbf{Answer:} Thank you for pointing this out. We included the discussions of the Reynolds
effects in the answer to the question above. As well, we added Reynolds numbers to the Table for
the capillary range regime and the grid depedency study. 

\textbf{Question:}
\begin{quotation}
In this work, the diphasic flow is driven by a body force. The body force is therefore the control
parameter in the simulations. The Authors should give more detail about this body force. Indeed,
although the estimation of the body force from the pressure gradient is well described, it is not
clear
where this body force appears in the governing equations (Equations 2, 3 and 4). Does this body
force correspond to the variable $F_i$ in Equation 2? Should not this force or its equivalent appear
in
the Navier-Stokes equations (Equation 4)?
\end{quotation}

\textbf{Answer:} We updated equation (4) to include the force contribution. As well, the definition
of $P_{\alpha\beta}$ is added. $F_i$ is the force population which corresponds to the force
inclusion $\bm{F}$. We added the citation which describes how the force population is formed and
how it
mimics the force behavior.

\textbf{Question:}
\begin{quotation}
I would suggest to calculate the Bond number to check if the shapes of the bubbles and the main
results of the paper are not influenced.
\end{quotation}

\textbf{Answer:} 
The maximum Bond number $Bo=\frac{\rho g N_y^2}{\gamma}$ in the
current simulation equals to $7$, which is less than the number for which the shape of the bubble is
affected according to \cite{zheng-large-ratio}.  

\textbf{Question:}
\begin{quotation}
The use of the variable $Ca_{lit}$ and the calculation of the corresponding pressure gradient yield
estimations of the body force to apply and of the film thickness to choose grid resolution. The
presentation
of the corresponding procedure is repeated twice in Section 5-1 for $Ca=0.005$ and $Ca=0.05$. It is
again described in a different way in Section 5-4. Similar equations for the calculation of the
pressure
gradient are thus given in Equations 6, 14 and 16. The Authors should summarize these equations
into one unless they explain their differences. In fact, it would probably be straightforward not to
detail the calculation of estimated variables since they are intermediary variables which allow the
design of the benchmark but do not influence directly the results. Also, the Authors could possibly
say they control and impose directly the body force without describing all the estimation procedure.
\end{quotation}

\textbf{Answer:} Thank you for the suggestion. We agree that there is redundant information
about initialization and this was taken into account by reducing repetitive text and
introducing step-by-step explations, see Section about initialization (lines 259-305). We also added
the
recommendations for controlling parameters in the same section.  The
dimensionalization of the problem and proper benchmarking is quite difficult in the lattice
Boltzmann framework. You can refer to the lbmethod.org/forum where half of all questions related to
the lattice Boltzmann method are refered to the non-dimensionalization procedure. Therefore, we
decided to
keep suggestions for controlling and initialization of simulations as informative guidelines for the
LBM community.

\textbf{Question:}
\begin{quotation}
 Moreover, the estimations of the pressure gradient for the initialisation of the scheme are
obtained
under the assumption of an axisymmetric Poiseuille flow. However, among the ”Key words”, one can
read the key word ”Flow between plates” in agreement with the 2D non axisymmetric simulations
that are done in this work. The Authors should discuss this choice of estimating the pressure
gradient with the axisymmetric assumption instead of a Hele-Shaw flow (factor 1/12 instead of 1/8
in Equation 5).
\end{quotation}

\textbf{Answer:} The factor $1/8$ is taken for the maximum velocity for the Poiseille flow between
plates. Our assumption that the bubble travels with the constant velocity which equals to the
maximum velocity of the Poiseuille profile between plates. Factor $1/12$ comes from the averaging
of the flow rate. However, bubble moves faster than the liquid, i.e. the average liquid velocity.
That's why the factor $1/8$ was chosen. The simulations show that factor $1/8$ works reasonably
well.

\textbf{Question:}
\begin{quotation}
The Section on LBM is concise and this seems a reasonable choice when publishing such a work in a
journal of applied science. The Authors would improve significantly the manuscript by defining more
rigorously some of the variables of the model. For instance, the Authors refer to surface tension in
lines 106, 111 and 123 via the parameter k, the weights w and the quantity 8kA/9 respectively.
A suggestion is to emphasize within an array the links between physical parameters and the model
parameters. The variables $i$, $f_i^{eq}$ , $g_i$ should be defined and the Equations for $f$ and $g$
in Equation
3 should be described (which equation corresponds to continuity equation...). Equation 2 could
perhaps be rewritten with the use of a generic variable in order to remove any ambiguity with the
physical variable f of Equation 3. The notation $i = 1 \div 8$ is unusual, it would be clearer to
use a
classical notation such as $1 \leq i \leq 8$. The sound speed $c_s^2$ should be defined just after
its
introduction in line 107.
\end{quotation}

\textbf{Answer:} We wrote an additional paragraph emphasizing the connection of the physical world
parameters with the lattice Botlzmann parameters (lines $149-160$). As well more explanations are
put towards the
interface surface tension description (lines $136-139$). The reference to the
continuity and the Navier-Stokes equations are added (lines $140-143$). The equation (2) is not
rewritten in terms of
generic variables as far as it's a historical LBM notation for the Navier-Stokes and phase
governing equations. The notations have been corrected.

\textbf{Question:}
\begin{quotation}
The name of the Section ”Lattice Boltzmann benchmark” lets the reader think that this Section
would describe the numerical procedure (assumptions, choice of the geometry and the grid, boundary
conditions, control parameters, initialization of the scheme). This is not the case because this
Section
develops the challenges and the difficulties of using the Lattice Boltzmann method to bubble flows.
This Section should be renamed or reorganized.
\end{quotation}

\textbf{Answer:} We renamed the section to ``Approach to benchmark''.

\textbf{Question:}
\begin{quotation}
The Authors should make a distinction between variables in lattice units and dimensional or non
dimensional variables. For instance, in line 195, the definition $H_{eff} = N_y - 2$ where $N_y$ is
the
number of nodes in the vertical direction indicates that $H_{eff}$ has lattice units. However, this
is in
contradiction with the use of $H_{eff}$ in Equation 5. The Authors should define each variable
without
any ambiguity on its units. Besides, the unit of $U_{bubble}$ should also be given. These
improvements
would fit well in the Subsection named ”The nondimensionalization and initialization procedure”
which does not in fact introduce much nondimensionalisation procedure.
\end{quotation}

\textbf{Answer:} We added a decription paragraph about the approach to units (lines 149-160). In
fact, all equations are the same in the
lattice Boltzmann framework or in a physical framework. One can substitute any quantity in
preferred units as soon as the capillary number is the same in the LBM and the physical world. 

\textbf{Question:}
\begin{quotation}
Some notations are unclear. For instance, the film thickness is denoted as $h_{\infty}$ in Equation
7 whereas
it is denoted $\delta$ in most of the paper. The notations w or $\delta_{sim}$ are also found before
line 381 or in
Table 3 respectively. The thickness of the interface is denoted $\xi$ in Figure 1 but the same
variable
is
denoted $5\xi$ in Table 1. The speed of the bubble is denoted $U_{\mathrm{bubble}}$ in some places
and $U_b$ in others.
The variable $Uin$ Equation 5 does not seem to be defined.
\end{quotation}

\textbf{Answer:} We thoroughly corrected all the inconsistencies.

\textbf{Question:}
\begin{quotation}
Three figures in the manuscript are scanned from the litterature (Ref. 3, 10 and 16). This is
unusual
for an article which is not a review. I encourage the Authors to remove at least the two first ones
which are not essential.
\end{quotation}

\textbf{Answer:} We eliminated unnecessary pictures and references to them.

\textbf{Question:} 
\begin{quotation}
In Figure 7, the points of data reported from the numerical work of Ref. 3 exhibit a surprisingly
noisy trend that corresponds more to an experimental study. The Authors should explain how they
obtain this data and plot them.
\end{quotation}

\textbf{Answer:} We added a description to the figure caption explaining data acquisition.

\textbf{Question:} 
\begin{quotation}
The plots of Figure 8 are of less quality than the other figures of the paper.
\end{quotation}

\textbf{Answer:} Unfortunately, it was not possible to extract data as a picture from the paper by
\citet{sehgal-microchannel}. Thus, the curves were extracted from the graphs and represented in a
new figure scaled to the present simulations for comparison reasons.

\textbf{Question:}
\begin{quotation}
The work of Yang seems to be close from the present study since it involves the
solution of Bretherton
flow with a Lattice Boltzmann method. The Authors should discuss this paper and emphasize the
differences between their work and this paper.
\end{quotation}

\textbf{Answer:} We outlined the differences in the introduction section (lines $92-100$).

\textbf{Question:} It would be interesting if the Authors could mention a study in which a level-set
method is used in
order to complete the bibliography.

\textbf{Answer:} We added references to the introduction section [17-19].

\textbf{Questions:} 
\begin{quotation}
A secondary quantity of interest is the bubble speed or in a non-dimensional
form the capillary
number. This quantity is measured at the center of the bubbles. The Authors should describe how
this quantity as well as the center of the bubble are measured.
\end{quotation}

\textbf{Answer:} Thank you for pointing that out. In the classical Bretherton problem
$U_{\mathrm{bubble}}$ is the bubble velocity, i.e. the interface velocity. We recalculate all the
results taking into the consideration the interface velocity at the bubble tip. Qualitatively, the
results are the same. 

\textbf{Question:} 
\begin{quotation}
 If possible, the Authors should present results showing that steady state is reached. It would
also
be interesting if the Authors show some patterns of the flow by plotting for instance the vector
field
of velocity around the bubbles.
\end{quotation}

\textbf{Answer:} We added a small steady-state study (lines $334-341$). We added an additional
paragraph which shows the change of flow pattern for the
streamlines as it is indicated in literature (lines $463-471$, Figure 8).

\textbf{Question:}
\begin{quotation}
In the Abstract, the Authors should give the range of real capillary numbers they measured instead
of the targeted values ($Ca_{lit}$).
\end{quotation}

\textbf{Answer:} We changed it to the measured capillary numbers.

\textbf{Question:} 
\begin{quotation}
The Authors should discuss more their results (Sections 5-4, 5-5, 5-6). Did the Authors measure the
pressure drops at the front and rear interfaces of the bubbles?
\end{quotation}

\textbf{Answer:} We added additional information about results to the corresponding sections. The
pressure drop is not measured as they are influenced largely by Reynolds number according to
\citet{heil-bretherton}.

\textbf{Question:}
\begin{quotation}
The Authors mention the shooting method. It is not clear if this method is really used or just
mentioned as a part of the discussion.
\end{quotation}

\textbf{Answer:} This is a part of discussion. The corresponding
sentence is added (lines $459-461$).

\textbf{Question:}
\begin{quotation}
It seems more appropriate to introduce the variables of the Bretherton’s problem as well as to plot
Figure 1 elsewhere than in the Section ”Results”.
\end{quotation}

\textbf{Answer:} The corresponding figure for the classical Bretherton problem is added. We
separated it from the numerical benchmark layout.

\textbf{Question:}
\begin{quotation}
Minor corrections...
\end{quotation}

\textbf{Answer:} All the corrections are taken into the account.

\bibliographystyle{unsrtnat}
\bibliography{answers}

\end{document}
